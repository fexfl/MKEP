\section*{Radiation}
$S_0=1360,5\frac{W}{m^2}$\\
$d\lambda B(\lambda,T)=\frac{2hc^2}{\lambda^5}\frac{1}{e^{\frac{hc}{\lambda k T}}-1}d\lambda$\\
$\lambda_{max}=\frac{2,898\cdot10^{-3}mK}{T}$\\
$F=\epsilon\sigma T^4$, $\sigma=5,67\cdot10^{-8}\frac{W}{m^2K^4}$\\
$\alpha + \rho + \tau = 1$, $\alpha = \epsilon$\\
Albedo of atmosphere + surface: $A = 30\%$
\subsection*{Global radiation balance} 
Without atmosphere: $T = \left(\frac{S_0(1-A)}{4\epsilon\sigma}\right)^{\frac{1}{4}}$\\
With atmosphere:\\
$\frac{S_0}{4}(1-A)+\epsilon_a\sigma T_a^4=\sigma T_S^4$\\
$\epsilon_a\sigma T_S^4=2\epsilon_a\sigma T_a^4$\\
$T_S = \left(\frac{S_0(1-A)}{2\sigma(2-\epsilon_a)}\right)^{\frac{1}{4}}$\\
$T_A = \left(\frac{S_0(1-A)}{4\sigma(2-\epsilon_a)}\right)^{\frac{1}{4}}$
\section*{Earths fluids}
\subsection*{Atmosphere}
Equation of state - air:\\
$pV=nRT$, $\rho = \frac{nM}{V}=\frac{pM}{RT}$\\
Pressure profile - atmosphere:\\
$p(z)=p_0 e^{-\frac{Mg}{RT}z}=p_0 e^{-\frac{z}{H}}$\\
Surface pressure: $p_0=1,013\cdot10^{5}Pa$\\
Surface density: $\rho_S = 1,235\frac{kg}{m^3}$\\
Kin. Visc.: $\nu=1,34\cdot10^{-5}\frac{m^2}{s}$\\
$c_p=1005\frac{J}{kgK}$, $c_V=718\frac{J}{kgK}$\\
Saturation Water Vapour Pressure:\\
$p_S(T)=p_0exp(-\frac{L}{R}(\frac{1}{T}-\frac{1}{T_0})))$\\
Magnus Formula:\\
$e_S=6,11hPa\cdot10^{\frac{7,5\cdot T[C]}{238 + T[C]}}$\\
Absolute Humidity: $a=\frac{m_V}{V}$\\
Specific Humidity: $q=\frac{\rho_V}{\rho_{air}}$\\
Relative Humidity: $f=\frac{e}{e_S}=\frac{\rho_V R_V T}{e_S}$\\
\subsection*{Ocean}
Pressure Profile - Ocean:\\
$p(z)=p_0-\rho g z$\\
Salinity: $S=\frac{\text{Mass of Salt}}{\text{Mass of Solution}}$\\
Density: $\rho=1\cdot10^3\frac{kg}{m^3}$\\
Kin. Visc.:$\nu=10^{-6}\frac{m^2}{s}$\\
Specific Heat: $c_W=4,18\cdot10^3\frac{J}{kgK}$\\
Equation of state - water:\\
$\rho(T,S,p)=\rho_0[1-\alpha(T-T_0)+\beta(S-S_0)+\gamma(p-p_0)]$, $\omega=(-1)^{\delta_{\omega,\alpha}}\frac{1}{\rho}\left(\frac{\partial \rho}{\partial A_\omega}\right)_{B,C}$
\section*{Diffusion and Advection}
Concentration: $c=\frac{m}{V} or \frac{n}{V} or \frac{N}{V}$\\
Flux Density: $j_x=\frac{1}{A}\frac{\Delta m_x}{\Delta t}$\\
Flux: $\Phi=\int\vec{j}d\vec{A}=-\frac{\partial m}{\partial t}$\\
Advection: $j_ad = \frac{1}{A}\frac{\partial N}{\partial t}=cv_x$, $\vec{j}_{ad}=c\vec{v}$\\
Diffusion: $\vec{j}_{dif}=-D\vec{\nabla} c$, $D=\frac{1}{3}\bar{v}\lambda$\\
Mean free path: $\lambda$\\
Mobility: $\gamma = \frac{\vec{v}}{\vec{F}_e}=-\frac{\vec{v}}{\vec{F}_{fric}}$\\
Stokes law: $\vec{F}_f=-6\pi\mu r\vec{v}$\\
Einstein relation: $j=vn-D\frac{\partial n}{\partial z}$ $\Rightarrow$\\ $D=\gamma kT$\\
Continuity equation: $\frac{\partial c}{\partial t}=-\vec{\nabla}\vec{j}$\\
Fick's laws: $\vec{j}=-D\vec{\nabla}c$, $\frac{\partial c}{\partial t} = D\Delta c$\\
Advection-Diffusion-equation:\\
$\frac{\partial c}{\partial t}=\left(\frac{\partial c}{\partial t}\right)_{ad}+\left(\frac{\partial c}{\partial t}\right)_{diff}=-\vec{v}\vec{\nabla}c+D\Delta c$\\
Heat conduction: $j_T = -\kappa\frac{\partial T}{\partial x}$, $\kappa = \frac{\lambda}{c_p\rho}$\\
Momentum diffusion: $\tau_{xz}=-\nu\frac{\partial}{\partial z}(\rho v_x)$\\
Diffusion constants:\\
$D_{air}=10^{-5}$, $D_{water}=10^{-9}$\\
$\kappa_{air}=10^{-5}$, $\kappa_{water}=10^{-7}$\\
For $\nu$ see above
\section*{Geophysical Fluid Dynamics}
Continuum hypothesis: Internal forces cancel, valid if $\frac{\lambda}{L}\ll 1$\\
$\tau=\frac{F_S}{A}=-\mu\frac{dv_x}{dy}=-\rho\nu\frac{dv_x}{dy}$\\
Newtonian Fluid: $\mu=$const.\\
Eulerian View: $\frac{\partial\psi}{\partial t}$\\
Lagrangian View: $\frac{d\psi}{dt}=\frac{\partial\psi}{\partial t} + (\vec{v}\vec{\nabla})\psi$\\
Incompressible fluid: $\frac{d\rho}{dt}=0 \Rightarrow \vec{\nabla}\vec{v}=0$\\
$\vec{f}=\frac{\vec{F}}{V}=\rho \frac{d\vec{v}}{dt}=\rho\frac{\partial\vec{v}}{\partial t}+\rho(\vec{v}\vec{\nabla})\vec{v}$
\subsection*{Forces in fluid media}
$\vec{f}_g=-\rho g \vec{e}_z = -\rho\vec{\nabla}\phi(z,\varphi)$\\
$\vec{f}_p=-\vec{\nabla}p$\\
$\vec{a}_f=\frac{1}{\rho}\vec{\nabla}\tau_{ij}=\nu\Delta \vec{v}$\\
$\tau_{ij}=-\mu\left(\frac{\partial v_i}{\partial x_j}+\frac{\partial v_j}{\partial x_i}\right)$\\
Earth Coordinate system: z upwards, y to the north, x to the east\\
$\left(\frac{d}{dt}\right)_{in}=\left(\frac{d}{dt}\right)_{rot}+(\vec{\Omega}\times)$\\
$\Omega=7,29\cdot10^{-5}Hz$\\
$\vec{a}_C=-2(\vec{\Omega}\times\vec{v})\approx -f(\vec{e}_z\times\vec{v})$, $f=2\Omega\sin{\varphi}$\\
Northern Hemisphere: $f>0$, deflection to the right.\\
Southern Hemisphere: $f<0$, deflection to the left.\\
\subsection*{Navier Stokes}
$\frac{d\vec{v}}{dt}=\frac{\partial\vec{v}}{\partial t}+(\vec{v}\vec{\nabla})\vec{v}=-\vec{g}-\frac{1}{\rho}\vec{\nabla}p-f(\vec{e}_z\times\vec{v})+\nu\Delta\vec{v}$\\
Scale Analysis:\\
$\frac{\partial\vec{v}}{\partial t},(\vec{v}\vec{\nabla})\vec{v}\propto \frac{U^2}{L}$\\
$-\frac{1}{\rho}\vec{\nabla}p\propto \frac{\Delta p}{\rho L}$\\
$f(\vec{e}_z\times\vec{v})\propto fU$\\
$\nu\Delta\vec{v}\propto \nu\frac{U}{L^2}$\\
$\text{Rossby} = \frac{\text{advection}}{\text{coriolis}}=\frac{U}{fL}$\\
$\text{Reynolds}=\frac{\text{advection}}{\text{friction}}=\frac{UL}{\nu}$\\
$\text{Euler}=\frac{\text{advection}}{\text{pressure}}=\frac{\rho U^2}{\Delta p}$\\
Horizontal components:
\subsection*{Initial currents}
Advection/Total derivative term and Coriolis term remaining\\
$\frac{d^2u}{dt^2}=-f^2u$ Harmonic oscillation
\subsection*{Geostrophic balance}
Pressure gradient and coriolis term remaining\\
$u=-\frac{1}{f\rho}\frac{\partial p}{\partial y}$\\
$v=\frac{1}{f\rho}\frac{\partial p}{\partial x}$\\
Geostrophic flow aligned with Isobars. Force balance between pressure gradient and coriolis.\\
Northern Hemisphere: Counterclockwise around L (Cyclone), clockwise around H (Anticyclone)\\
Southern Hemisphere: Clockwise around L, counterclockwise around H\\
With friction: reduces coriolis force.\\
Northern Hemisphere: convergence at cyclone, divergence at anticyclone.
\subsection*{Thermal wind}
Horizontal temperature difference -> baroclinic\\
Surface flow from cold to warm, high altitude flow from warm to cold. Ascending at warm side.\\
With coriolis term:\\
$\frac{\partial u}{\partial z}=-\frac{g}{fT}\frac{\partial \bar{T}}{\partial y}$\\
Linear increase in wind velocity with altitude\\
Vertical component:
\subsection*{Hydrostatic equilibrium}
Gravity and Pressure gradient term remaining\\
$\frac{\partial p}{\partial z}=-\rho g$\\
Buoyancy:\\
$F_G=-g\rho_pV_p$, $F_B=g\rho_EV_p$\\
$F=-g(\rho_p-\rho_E)V_p$\\
$b\equiv \frac{F}{\rho_p V_p}=-g\frac{\rho_p-\rho_E}{\rho_p}=a_z$
\subsection*{Stability of Stratification}
$b\approx\frac{g}{\rho}\left(\frac{\partial\rho}{\partial z}\right)_E \Delta z$\\
If $\frac{\partial\rho}{\partial z}>0$ unstable\\
If $\frac{\partial\rho}{\partial z}=0$ neutral\\
If $\frac{\partial\rho}{\partial z}<0$ stable\\
$\ddot{\xi}=b=\frac{g}{\rho}\frac{\partial\rho}{\partial z} \xi$\\
$N^2=-\frac{g}{\rho}\frac{\partial\rho}{\partial z}$\\
If $\rho$ only depends on $T$: $N^2\propto \frac{dT}{dz}$\\
$N^2>0$ stable, $N^2<0$ unstable\\
Adiabatic temperature changes:\\
$N^2=\frac{g}{T_E}\left(\frac{dT_E}{dz}-\frac{dT_p}{dz}\right)$\\
Dry adiabatic lapse rate:\\ $\frac{dT_p}{dz}=-\frac{g}{c_p}=-\Gamma_d=-\frac{1K}{100m}$\\
Moist adiabatic lapse rate: $\Gamma_m=-6,5\frac{K}{km}$\\
$\Theta=T\left(\frac{p_0}{p}\right)^{\frac{\kappa-1}{\kappa}}$, $\kappa = 7/5$\\
Temperature that an air parcel reaches if it is adiabatically compressed to $p_0$.\\
$N^2 = \frac{g}{\Theta}\frac{d\Theta}{dz}$\\
$\frac{d\Theta}{dz}=\frac{\Theta}{T}\left(\frac{dT}{dz}+\Gamma_d\right)$
\subsection*{Vorticity}
$\vec{\zeta}=\vec{\nabla}\times\vec{v}$\\
Curl in cylindrical coordinates:\\
$\vec{\nabla}\times\vec{F}=
\begin{pmatrix}
\frac{1}{r}\frac{\partial F_z}{\partial \varphi}-\frac{\partial F_\varphi}{\partial z}\\
\frac{\partial F_r}{\partial z}-\frac{\partial F_z}{\partial r}\\
\frac{1}{r}\frac{\partial}{\partial r}(rF_\varphi)-\frac{1}{r}\frac{\partial F_r}{\partial \varphi}
\end{pmatrix}$\\
$ C = \oint \vec{v}d\vec{l}$, $\frac{dC}{dA_z}=\zeta_z$\\
Rigid Rotator: $\vec{v}=\omega r\vec{e}_\varphi$\\
$\zeta_z = 2\omega$\\
Shear flow:\\
$\zeta_z = \frac{\partial v}{\partial x}-\frac{\partial u}{\partial y}$\\
Irrotational flow: $v_\varphi=\frac{\alpha}{r}$\\
$\zeta_z=0$, Parcel stays orientated in the same direction\\
Earth: $\zeta_z=2\Omega_z=f$\\
Circulation theorem: If invicid ($\nu=0$) and barotropic: $\frac{dC}{dt}=0$
\subsection*{Absolute Vorticity}
$\eta = \zeta + f$\\
Vorticity equation:\\
$\frac{\partial\eta}{\partial t}+\eta(\vec{\nabla}_h\vec{v})+\vec{v}(\vec{\nabla}_h\eta)=(\vec{\nabla}\times\frac{1}{\rho}\vec{f}_{ext})_z$\\
Potential Vorticity (moving water column):\\
$\Pi=\frac{\eta}{H}$ is conserved.\\
$\Rightarrow$ Rossby waves: Changes in $f \rightarrow$ change $\zeta \rightarrow$ higher $f \rightarrow$ again
\subsection*{Turbulence}
Reynolds decomposition: $\vec{v}=\bar{\vec{v}}+\vec{v}'$\\
$\bar{\vec{v}}=\frac{1}{T}\int_0^T\vec{v}dt$\\
$\varepsilon_{kin}=\frac{1}{2}\rho\bar{v}^2$\\
$\text{TKE}=\frac{1}{2}\bar{v'^2}$\\
$\bar{f'}=0$, $\bar{\bar{f}}=0$, $\bar{\bar{f}g'}=0$, $\bar{fg}=\bar{f}\bar{g}+\bar{f'g'}$\\
Continuity equation: $\vec{\nabla}\bar{\vec{v}}=0$, $\vec{\nabla}\vec{v}'=0$\\
From Navier Stokes:\\
$\bar{v_i'v_j'}=-K_{ij}\left(\frac{\partial\bar{v_i}}{\partial x_j}-\frac{\partial\bar{v_j}}{\partial x_i}\right)$\\
$K_{ij}$ eddy turbulent viscosity\\
Reynolds stress:\\
$\tau_{xz,turb}=\rho\bar{v_i'v_j'}=-\rho K_z\frac{\partial\bar{v_x}}{\partial z}$\\
Frequency spectrum of turbulence:\\
$F(\nu)=\int_{-\infty}^{\infty}v'(t)e^{-2\pi i\nu t}dt$\\
$dE = \bar{v'^2}f(\nu)d\nu$\\
Kolmogorov:\\
$\varepsilon = -\frac{d\text{TKE}}{dt}$\\
Scale defined by $\nu [\frac{L^2}{T}]$ and $\varepsilon [\frac{L^2}{T^3}]$\\
Autocorrelation:\\
$R_L(\tau)=\frac{\braket{v'(t)v'(t-\tau)}}{\braket{v'^2(t)}}$\\
$R_E(\tau)=\frac{\bar{v'(t)v'(t-\tau)}}{\bar{v'^2(t)}}$\\
$R_E(\tau)=\mathcal{F}(f(\nu))$\\
Turbulent Mixing - Theorem of Taylor:\\
$\sigma_x^2(t)=2\bar{v_x'^2(t)}\int_0^t\int_0^{t'}R_{L,x}(\tau)d\tau dt'$\\
For large $\tau$: $\int_0^{t'}R_{L,x}(\tau)d\tau=\tau_L=\text{const.}$\\
$t\ll\tau_L$: $R_L=1$, $\sigma_x(t)=\sqrt{\bar{v'^2}}t$\\
$t\gg\tau_L$: $R_L=0$, $\sigma_x(t)=\sqrt{2\bar{v'^2}\tau_L t}$\\
Molecular vs Turbulent Diffusion:\\
Molecular: $\sigma^2=\frac{2}{3}v\lambda t$\\
Turbulent: see above\\
Mixing Length: $l'=v'\tau_L$\\
$\sigma^2=2\braket{v'l'}t$\\
Diffusion Coeff.: $K_x \approx K_y > K_z \gg D$
\subsection*{Boundary Layers}
\textit{Free Atmosphere:} $\tau\approx 0$\\
\textit{Ekman Layer:} $\frac{\partial u}{\partial z}$ and $\tau$ decreasing with height\\
$\frac{d \tau}{dz}=f\rho(\vec{v}-\vec{v}_g)\vec{e}_z$\\
\textit{Prandtl Layer:} Turbulent friction, $\tau\approx\text{const.}$\\
Increasing eddies, energy cascade\\
\textit{Molecular viscous layer:} $\tau=-\rho\nu\frac{\partial u}{\partial z}=\text{const.}$\\
\subsection*{Prandtl Mixing Length approach}
Mixing length increases linearly with height: $l_z'=\kappa z$, $\kappa=0,41$\\
$u'=-l_z'\frac{\partial u}{\partial z}$\\
$|u'|\approx|w'|\approx|l_z'\frac{\partial u}{\partial z}|$\\
$\tau_{xz}=\rho\bar{u'w'}=\rho\kappa^2 z^2\left(\frac{\partial u}{\partial z}\right)$\\
$u_*=\sqrt{\frac{|\tau|}{\rho}}$\\
$\frac{\partial u}{\partial z}=\frac{u_*}{\kappa z}$\\
Momentum diffusion:\\
$\tau_{xz}=-K_z\frac{\partial}{\partial z}(\rho u) \Rightarrow K_z=\kappa u_* z$
\subsection*{Logarithmic wind profile}
From the differential equation above:\\
$u(z)=\frac{u_*}{\kappa}\ln{\frac{z}{z_0}}$, $z_0$ surface roughness\\
$u(z)=u_r\frac{\ln{\left(\frac{z}{z_0}\right)}}{\ln{\left(\frac{z_r}{z_0}\right)}}$\\
$u_*=u_r\frac{\kappa}{\ln{\left(\frac{z_r}{z_0}\right)}}$\\
$z_0\approx\frac{\nu}{9u_*}$\\
$K_z = \kappa u_* z = \frac{u_r\kappa^2 z}{\ln{\left(\frac{z_r}{z_0}\right)}}$\\
Drag coefficient: $c_D = \left(\frac{u_*}{u_r}\right)^2$\\
\subsection*{Wind profile in the PBL}
Free atmosphere: geostrophic\\
Ekman Layer: ageostrophic, Ekman Spiral\\
Prandtl Layer: logarithmic
\subsection*{Wind-driven surface currents}
Air: $\tau_{xz}(0)=\rho_{air}c_p u_{10}^2$\\
Water: $\tau_{xz}(z)=\rho_{water}K_z\frac{\partial u}{\partial z}$\\
$\frac{\partial u}{\partial z}\big|_{z=0}=\frac{\rho_{air}c_p u_{10}^2}{\rho_{water}K_z}$\\
\subsection*{Equation of motion for surface currents}
Geostrophic with friction:\\
$0=-\frac{1}{\rho}\vec{\nabla}_hp-f(\vec{e}_z\times\vec{v}_h)+K_z\frac{\partial^2\vec{v}_h}{\partial z^2}$\\
Seperate:\\
Geostrophic: $f(\vec{e}_z\times\vec{v}_g)=-\frac{1}{\rho}\vec{\nabla}_hp$\\
Ageostrophic: $f(\vec{e}_z\times\vec{v}_{ag})=K_z\frac{\partial^2 \vec{v}_h}{\partial z^2}$\\
$\frac{\partial^2 v}{\partial z^2}=\frac{f}{K_z}u$\\
$\frac{\partial^2 u}{\partial z^2}=-\frac{f}{K_z}v$\\
$u(z)=u_{0}e^{bz}\cos{(bz-\frac{\pi}{4})}$\\
$v(z)=v_{0}e^{bz}\sin{(bz-\frac{\pi}{4})}$, $b=\sqrt{\frac{f}{2K_z}}$\\
Current 45 degrees to the right of the wind\\
Mass transport: To the right of the wind\\
$\vec{M}_{Ek}=\int_{-\delta}^0\rho v\,dz=\frac{\vec{\tau}_{wind}\times\vec{e}_z}{f}$
\section*{Global Ocean circulation}
At surface: turbulent mixing, wind driven turbulence\\
Below: Thermocline, dominated by density gradients
\subsection*{Ekman Pumping}
Northern Hemisphere:\\
Counter-clockwise around cyclone $\Rightarrow$ Mass transport causes surface divergence and upwelling\\
Clockwise around anticyclone $\Rightarrow$ Mass transport causes surface convergence and downwelling\\
$w_{Ek}=\frac{1}{\rho}\vec{e}_z(\vec{\nabla}\times\frac{\vec{\tau}_{wind}}{f})$
\subsection*{Effect of Ekman pumping on the interior}
Below Ekman-Layer: geostrophic $u_g$, $v_g$\\
$\vec{\nabla}_h\vec{v}_g=-\frac{\beta}{f}v_g$\\
$\beta v_g=f\frac{\partial w}{\partial z}$, $\beta = \frac{\partial f}{\partial y}=\frac{2\Omega\cos{\varphi}}{R}$\\
$(\beta(90)=0$, $\beta(75)=0,593$, $\beta(60)=1,145$, $\beta(45)=1,619$, $\beta(30)=1,983$, $\beta(0)=2,29)\cdot10^{-11}\frac{1}{ms}$\\
Southward flow below area of downward Ekman pumping
\subsection*{Sverdrup Theory}
$f\frac{\partial w}{\partial z}=\beta v-\frac{1}{\rho}\frac{\partial}{\partial z}\left(\frac{\partial\tau_y}{\partial x}-\frac{\partial\tau_x}{\partial y}\right)$\\
$\beta V=\frac{1}{\rho}\vec{e}_z(\vec{\nabla}\times\vec{\tau}_{wind})$\\
$V=\int_{-D}^0 v dz$
\section*{Global atmosphere circulation}
\subsection*{Direct thermal circulation}
Air rises on warm land $\Rightarrow$ low pressure on land, high pressure above\\
Air sinks on cold ocean $\Rightarrow$ high pressure on ocean, low pressure above\\
\subsection*{Thermal geostrophic wind}
With coriolis $\Rightarrow$ westerly wind\\
$v_x(z)=v_x(0)-\frac{g}{f\bar{T}}\frac{\partial \bar{T}}{\partial y}z$\\
Increasing wind speed with altitude due to increasing pressure difference
\subsection*{The Hadley cell}
From equator to 25 degrees: Hot and wet air rises at the equator, goes north, dry air sinks at 25 degrees. $\Rightarrow$ Intertropical Convergence Zone
\subsection*{Extratropical Circulation}
Three convection cells between equator and pole\\
North-/Southeast Trades, above Westerlies, above Polar easterlies
\subsection*{Rossby Waves}
$\frac{d\eta}{dt}=0\Rightarrow\frac{d\zeta}{dt}=-\beta v$\\
Initial pertubation leads to an oscillation in y-Direction\\
$k\omega-uk^2=-\beta$, $c=u-\frac{\beta \lambda^2}{4\pi^2}$\\
For stationary $c=0$: $\lambda\approx 5900km$
\section*{Groundwater and Soil physics}
Porosity: $\phi=\frac{V_{pores}}{V}$\\
Water content: $\theta = \frac{V_{water}}{V}$\\
Water flow in pores: Gravity, Pressure gradient remaining in vertical, pressure gradient and friction in horizontal\\
$=0-g-\frac{1}{\rho}\frac{\partial p}{\partial z} \Rightarrow p_{hyd}=p_0-\rho g z$\\
$p_d = p-p_{hyd}$\\
Hydraulic head: $h=\frac{p_d}{\rho g}=\frac{p}{\rho g}-z$\\
Horizontal: $\vec{\nabla}p-\rho\vec{g}=\mu\Delta\vec{v}$\\
Energy at point z: $E=pV+mgz$, $E=hmg$
\subsection*{Aquifer types}
Unconfined aquifer: $h=$ height of groundwater level\\
Confinded aquifer: $h>$ height of groundwater level
\subsection*{Darcy's law}
Filtergeschwindigkeit: $q=\frac{Q}{A}=-K\frac{\Delta h}{\Delta l}$
\subsection*{Laminar flow through a horizontal tube}
$\frac{1}{\rho}\frac{\partial p}{\partial x}=2\nu\frac{\partial^2 u}{\partial r^2}$\\
$u(r)=-\frac{1}{4\mu}\frac{\partial p}{\partial x}(R^2-r^2)$
\subsection*{Hagen-Poiseuille Law}
Integration over cross-section:\\
$Q=-\frac{\pi R^4}{8\mu}\frac{\partial p}{\partial x}$\\
Bundle of capillaries: $\phi = \frac{N\pi R^2}{A_{tot}}$\\
$q=\frac{Q}{A_{tot}}=-\frac{\phi R^2g}{8\nu}\frac{\partial h}{\partial x}$\\
Darcy: $K=\frac{\phi R^2 g}{8\nu}$
\subsection*{Flow equation for Groundwater}
$\vec{\nabla}\vec{q}=-S_s\frac{\partial h}{\partial t}-J_W$\\
$S_S=-\frac{1}{V_{tot}}\frac{\partial V_W}{\partial h}$\\
Special case: $K$ const, no $J_W$:\\
$\frac{\partial h}{\partial t}=\alpha \Delta h$ Diffusion for hydraulic head
\section*{Cryosphere}
Glaciers: flow driven by slope of plane\\
Ice sheets: flow driven by slope of ice surface\\
Stress-Strain: $\varepsilon=\lim_{l\rightarrow0}\frac{\Delta l}{l}$\\
$\Delta l=l_x\frac{\partial u}{\partial x}dt$\\
$\dot{\varepsilon}_{xy}=\frac{1}{2}\left(\frac{\partial u}{\partial y}+\frac{\partial v}{\partial x}\right)$\\
\textit{Solids:} $\sigma = E\varepsilon$, $\dot{\varepsilon}=0$\\
\textit{Fluids:} $\tau_{ij}=-2\mu\dot{\varepsilon}_{ij}$\\
\textit{Ice:} $\dot{\varepsilon}_{ij}=A|\tau_{eff}^{n-1}|\tau_{ij}$, $\mu^*=\frac{1}{A|\tau_{eff}^{n-1}|}$, $n\approx 3$\\
\subsection*{Shear stress and shape of glaciers}
$\tau < \tau_0$ yield stress\\
$h\approx \frac{\tau_0}{\rho g}\frac{1}{\sin{\alpha}}$, $\tau_0=1$, $\frac{\tau_0}{\rho g}\approx 11m$\\
$H^2=\frac{\tau_0}{\rho g}2L$, $V\propto H^5 = L^2 H$
\subsection*{Kinematic ice sheet flow model}
$\frac{\partial H}{\partial t}=-\frac{\partial}{\partial x}(v_x H)+A$\\
At $x=0$, stationary, continuity:\\
$v_z(z)=A(1-\frac{z}{H})$\\
$t(z)=-\frac{H}{A}\ln{(1-\frac{z}{H})}$
\section*{Isotopes}
$A$ Massenzahl, $Z$ Kernladungszahl, $N$ Neutronenzahl\\
Alpha decay: $\ce{_Z^AX_N}\rightarrow \ce{_{Z-2}^{A-4}Y_{N-2}}+\alpha$\\
Beta Minus decay:\\
$\ce{_{Z}^AX_N}\rightarrow\ce{_{Z+1}^AY_{N-1}}+\bar{\nu}_e+e^-$\\
Radioisotopes:\\
-Primordial (solar system)\\
-Radiogenic (from decay)\\
-Cosmogenic (cosmic rays)\\
-Anthropogenic (technical processes)\\
Radioactive decay: $N(t)=N_0e^{-\lambda t}$
\subsection*{Stable Isotopes}
$R=\frac{\text{rare}}{\text{common}}$\\
$\delta = \frac{R_{sample}}{R_{standard}}-1$\\
Fractionation:\\
$\alpha_{B/A}=\frac{R_B}{R_A}$, $\varepsilon=\alpha -1$\\
$\varepsilon>0$: Enrichment of rare, $\varepsilon<0$: Depletion of rare\\
$\delta_B\approx\delta_A+\varepsilon_{B/A}$ for $A\rightarrow B$\\
$R$ and $\alpha$ multiplicative, $\delta$ and $\varepsilon$ approx. additive\\
$R_A=R_{st}(1+\delta_A)$\\
$\alpha_{B/A}=\frac{1+\delta_B}{1+\delta_A}\approx 1+\delta_B-\delta_A$\\
$\varepsilon_{B/A}\approx \delta_B-\delta-A$\\
$\varepsilon_{B/A}\approx -\varepsilon_{A/B}$\\
Rule of thumb: $\ce{^2H}/\ce{^1H}$: $\frac{\Delta m}{m_l}=1$\\
$\ce{^{18}O}/\ce{^{16}O}$: $\frac{\Delta m}{m_l}=\frac{2}{16}=1/8$\\
$\varepsilon(\ce{^2H})\approx 8\varepsilon(\ce{^{18}O})$\\
Fractionation decreases with increasing temperature\\
Heavy isotopes are generally enriched in the denser phase
\subsection*{The Rayleigh process}
$\frac{R}{R_0}=\left(\frac{N}{N_0}\right)^{\alpha-1}\equiv f^{\alpha-1}$